\documentclass[openany]{mgr} % TODO change options at the end

\usepackage[T1]{fontenc}
%\usepackage{ae,aecompl} % TODO maybe this one looks better?
\usepackage{pslatex}
\usepackage[utf8]{inputenc}
\usepackage[english]{babel}
\usepackage[nottoc,notlot,notlof]{tocbibind}
\usepackage{graphicx}
\usepackage{url}
\urlstyle{same} % line breaks in URLs
% \usepackage[justification=centering]{caption} % TODO uncomment
\usepackage{amsthm}
\usepackage{mathtools}
% \usepackage[bookmarks=true, breaklinks]{hyperref} % TODO uncomment this and below 
% \usepackage[all]{hypcap} % needed to help hyperlinks direct correctly;

\newcommand{\includeimage}[2]{
\begin{figure}
\centering
\includegraphics[width=\textwidth]{img/#1}
\caption{#2}
\label{result:#1}
\end{figure}
}

\author{Michał Kowalski} % TODO inż.?
\supervisor{dr inż. Marek Woda} % TODO przecinek? jednostka?
\field{Informatyka (INF)}
\specialisation{Internet Engineering (INE)}
\title{Smartfon z systemem Android\\jako wysokopoziomowy sterownik robota}
\engtitle{Android smartphone\\as a high-level controller of a robot}

\begin{document}
\maketitle
\tableofcontents

\chapter{Introduction}

\section{Description of problem}

\section{Goal of a project}

\section{State of art}

\includeimage{SoA1}{robot1, \cite{SoA1} fdfsdfsdf dgfdgdf  gfhgfh }
\includeimage{SoA2}{robot2, \cite{SoA2}}
\includeimage{SoA3}{robot3, \cite{SoA3}}
\includeimage{SoA0}{own robot}

\chapter{Platforms}
\section{Android}
\section{MCU}

\chapter{Communication}

\section{Introduction}

\section{Communication through USB cable - MCU}
\subsection{UART}
\subsection{CDC}

\section{Communication through USB cable - Android}
Three ways to communicate over USB were found:
\begin{itemize}
  \item USB Host API \cite{android_reference},
  \item usb-serial-for-android library by mik3y \cite{mik3y},
  \item UsbSerial by felHR85 \cite{felHR85}.
\end{itemize}

\subsection{USB Host API}
\subsection{mik3y}
\subsection{felHR85}

\section{Summary}
% TODO performance and compatibility comparison

\chapter{Sensors}

\section{Introduction}
Modern smartphones has many sensors, and most of them can extend robot's
functionality. Sensors differ between phones, and new (or more advanced) ones can
be connected using possible connections (mostly USB and Bluetooth).
Most popular ones are:
\begin{itemize}
  \item touch screen,
  \item accelerometer,
  \item gyroscope,
  \item microphone(s),
  \item front and rear camera(s),
  \item position sensors:
  \begin{itemize}
    \item GPS,
    \item multilateration based on GSM and/or WiFi,
  \end{itemize}
  \item magnetometer,
  \item light sensor,
  \item proximity sensor.
\end{itemize}
Some (mostly high-end, or specialized ones) have also sensors like electronic
compass, humidity/temperature sensors, fingerprint scanner, or even thermal
camera.

\section{Face detection}
Available implementations of face detection includes:
\begin{itemize}
  \item FaceDetector API,
  \item Camera API,
  \item openCV for Android,
  \item openCV NDK.
\end{itemize}

% TODO short description + code samples

\subsection{FaceDetector API}
% hard to obtain bitmap from camera

\subsection{Camera API}
% max 5 faces
% 30 ms in good light/on screen, 60 ms in bad
% resolution doesn't have impact on speed, only on quality
% detection in background

\subsection{openCV for Android}
% RGB to grayscale < 1 ms
% 300ms-1400ms @ 960x720 NO THREADS
% 50 ms @ 144x176 NO THREADS
% 200 ms 240x320 NO THREADS / 1 THREAD
% detection in user's code

\subsection{openCV NDK}
% real time

\section{Summary}
% TODO performance comparison

\chapter{Summary}
% TODO summary of whole thesis

\begin{thebibliography}{9}
% TODO short URLs?

\bibitem{SoA1}
Ganeev Singh, Mobile Controlled Robot,\\
\url{engineersgarage.com/contribution/mobile-controlled-robot},
\date{29.05.2016}

\bibitem{SoA2}
Robotics Bible, Mobile Controlled Robot via GSM,\\
\url{roboticsbible.com/project-mobile-controlled-robot-without-microcontroller.html},
\date{29.05.2016}

\bibitem{SoA3}
Mayoogh Girish, Mobile Controlled Robot,\\
\url{diyhacking.com/mobile-controlled-robot},
\date{29.05.2016}

\bibitem{android_reference} 
Android Reference,
\url{developer.android.com}

\bibitem{mcu_on_eclipse} 
Erich Styger, MCU on Eclipse,
\url{mcuoneclipse.com/},
29.04.2016

\bibitem{opencv_reference} 
openCV Reference,
\url{http://opencv.org/}

\bibitem{stack_overflow} 
Stack Overflow,
\url{http://stackoverflow.com/}

\bibitem{mik3y} 
Mike Wakerly, usb-serial-for-android,\\
\url{github.com/mik3y/usb-serial-for-android},
20.05.2016

\bibitem{felHR85} 
Felipe Herranz, UsbSerial,
\url{github.com/felHR85/UsbSerial},
20.05.2016

\end{thebibliography}
\end{document}
