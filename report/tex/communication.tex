\chapter{Communication}

\section{Introduction}
Three ways to communicate over USB were found:
\begin{itemize}
  \item USB Host API \cite{android_reference},
  \item usb-serial-for-android library by mik3y \cite{mik3y},
  \item UsbSerial by felHR85 \cite{felHR85}.
\end{itemize}
Because of similar names of projects, they will be referenced as Host API,
mik3y and felHR85.

% TODO all screenshots on xperia
% TODO mention ``\0''

\section{USB Host API}

\lstinputlisting[float, label=lst:usbhostapi, caption=USB Host API]
{../code_samples/bulktransfer_sample.java}

\includeimage{comm/bulk/bulk_1}{USB Host API - sending 1 letter}
[\\Received text is correct. 
As it can be seen, in this test it takes about 22 ms to get an echo of each sent
message.
Although, during other (not documented with screenshot) tests, sometimes (but
less often) average duration was about 12 ms.]

\includeimage{comm/bulk/bulk_32}{USB Host API - sending 32 letters}
[\\Received text is correct. 
Average time of execution was 12 ms, but in other tests, it also happened (but
less often) to be 22 ms.]

\includeimage{comm/bulk/bulk_50}{USB Host API - sending 50 letters}
[\\Received text is correct. 
Average time of execution was 12 ms, but in opposition to sending 1 or 32
letters, other values were not observed.]

\includeimage{comm/bulk/bulk_64}{USB Host API - sending 64 letters}
[\\Received text is NOT correct. 
Only first message is received completely, but sent from MCU as two.
Next ones are missing last character, and have last non-missed character from
previous one at the beginning.]

\clearpage

\section{USB Serial by mik3y}

\lstinputlisting[float, label=lst:mik3y, caption=mik3y]
{../code_samples/mik3y_sample.java}

\clearpage

\section{USB Serial by felHR85}

\subsection{Asynchronous mode}

\lstinputlisting[float, label=lst:async, caption=felHR85 async]
{../code_samples/felhr85async_sample.java}

\includeimage{comm/async/async_cdc_1}{felHR85, async-CDC, 1 letter}
[\\Received text is correct, but in most cases, two messages were sent at once.
Average execution time was 11 ms.]

\includeimage{comm/async/async_cdc_32}{felHR85, async-CDC, 32 letters}
[\\Received text is correct. 
In this case, all messages were sent separately.
Average execution time was also 11 ms.]

% TODO test for 31 long?

\includeimage{comm/async/async_cdc_50}{felHR85, async-CDC, 50 letters}
[\\Received text is correct, and all messages were sent separately.
Each text is divided into two lines, because it doesn't fit the window, and
Android does that automatically.
Average execution time was 12 ms.]

\includeimage{comm/async/async_cdc_64}{felHR85, async-CDC, 64 letters}
[\\Received text is  NOT correct.
Result looks the same as for using USB Host API.]

\includeimage{comm/async/async_cdc_70}{felHR85, async-CDC, 70 letters}
[\\Received text is also not correct.
It can be also seen, that the same as for 64 chars long message, ending is
missing, and next message starts with first of missing letters.]

\includeimage{comm/async/async_uart_1}{felHR85, async-UART, 1 letter}
[\\Received text was correct, but send as only a few messages. 
In async-CDC it was only 2 messages at once, in this case all is send in 2-3.
Average execution time was about 1.5 ms, which is really faster than using CDC.]

\includeimage{comm/async/async_uart_32}{felHR85, async-UART, 32 letters}
[\\Received text looks messy, but is also regular, so it still can be seen,
that is correct. ``uart`` can be seen only once, so it means, that MCU can
continuously get text from buffer, before escaping loop.
Average execution time was about 34 ms, which is longer than using CDC, but
seems to be correlated to length of message.]

\includeimage{comm/async/async_uart_50}{felHR85, async-UART, 50 letters}
[\\Received text is correct, and looks neater than the one for 32 chars, but
it's only because of displaying it. 
As for 32 chars, ``uart: `` can be seen only from time to time (next one was
off the screen).
Average execution time was about 52 ms.]

\includeimage{comm/async/async_uart_64}{felHR85, async-UART, 64 letters}
[\\It's the first time, that 64 long message was transmitted completely without
any errors.
Average execution time was about 67 ms.]

\includeimage{comm/async/async_uart_70}{felHR85, async-UART, 70 letters}
[\\Because 64 chars long message was transmitted without errors, it was worth
checking, if even longer messages could be send. 
And indeed, with UART it's possible to do it without errors.
Average execution time was about 73 ms, so it actually increases with length.]

\clearpage

\subsection{Synchronous mode}

\lstinputlisting[float, label=lst:sync, caption=felHR85 sync]
{../code_samples/felhr85sync_sample.java}

\includeimage{comm/sync/sync_cdc_1}{felHR85, sync-CDC, 1 letter}
[]

\includeimage{comm/sync/sync_cdc_32}{felHR85, sync-CDC, 32 letters}
[]

\includeimage{comm/sync/sync_cdc_50}{felHR85, sync-CDC, 50 letters}
[]

\includeimage{comm/sync/sync_cdc_64}{felHR85, sync-CDC, 64 letters}
[]

\includeimage{comm/sync/sync_uart_1}{felHR85, sync-UART, 1 letter}
[]

\includeimage{comm/sync/sync_uart_32}{felHR85, sync-UART, 32 letters (1)}
[]

\includeimage{comm/sync/sync_uart_32_2}{felHR85, sync-UART, 32 letters (2)}
[]

\includeimage{comm/sync/sync_uart_50}{felHR85, sync-UART, 50 letters (1)}
[]

\includeimage{comm/sync/sync_uart_50_2}{felHR85, sync-UART, 50 letters (2)}
[]

\includeimage{comm/sync/sync_uart_64}{felHR85, sync-UART, 64 letters}
[]

\includeimage{comm/sync/sync_uart_70}{felHR85, sync-UART, 70 letters (1)}
[]

\includeimage{comm/sync/sync_uart_70_2}{felHR85, sync-UART, 70 letters (2)}
[]

\clearpage

\subsection{Division of messages}
% TODO division of messages

\includeimage{comm/div/div_async_cdc}{Division of message: async-CDC}
[\\Text is received in one message. 
Semicolon is visible on line with next message, because MCU sends new line
character (``\textbackslash n''), and semicolor is added on Android side after that.]

\includeimage{comm/div/div_async_cdc2}{Division of message: async-CDC with too
long message}
[\\It can be seen, what happens when message exceeds buffer.
Each message is received as two:
64 chars long, and 5 chars long (``5432'' + `` \textbackslash n'').]

\includeimage{comm/div/div_async_uart}{Division of message: async-UART}
[\\Message is usually send letter by letter, with few exceptions.
As it was noted earlier, ``uart'' occurs only from time to time, which means,
that MCU has something in buffer all the time.]

\includeimage{comm/div/div_sync_uart}{Division of message: sync-UART}
[\\MCU still reads and sends message char-by-char, but smartphone reads them in
blocks. 
It can also be seen, that all blocks are correct, and errors can be
seen only between them. 
Also in this case, ``uart`` can be seen only from time to time, but more
frequently than in ``sync'' version.]

% TODO sync-cdc?

\clearpage

\section{Summary}
% TODO pros/cons comparison

\begin{center}
\begin{tabular}{r|r|r|r|r|r|r|r|r|r|r|r}
& \multicolumn{3}{c|}{Xperia Neo - CDC} & \multicolumn{3}{c|}{Moto G - CDC} &
\multicolumn{5}{c}{Moto G - UART} \\
message length & 1 & 32 & \hspace{3ex}50 & 
1 & 32 & \hspace{3ex}50 & \hspace{1ex}1 & 
32 & 50 & 64 & 70
\\
\hline
USB Host API & 
11/22 & 12/22 & 12 & 11/22 & 12/22 & 12 &
\multicolumn{5}{c}{\cellcolor{red!50}not working}
\\
felHR85 - async & 
11 & 11 & 12 & 11 & 11 & 12 & 2 & 34 & 52 & 67 & 73 \\
felHR85 - sync & 
25 & 21 & 24 & 22 & 12 & 12 &
4 & 33 & \cellcolor{yellow!50}57 & 66 & \cellcolor{yellow!50}80
\end{tabular}
\end{center}