\chapter{Communication}

\section{Introduction}
The most important thing in extending robot's capabilities with a~smartphone, is
to connect smartphone with MCU.
Because chosen MCU supports USB CDC and UART through it's OpenSDA USB port, one
of possible ways is to do it with USB cable connected to smartphone's micro USB port.
Next thing is to find a~way, which will allow for access to a~port, and to send
a text message through it.
After research, three ways were found:
\begin{itemize}
  \item USB Host API \cite{android_reference},
  \item usb-serial-for-android library by mik3y \cite{mik3y},
  \item UsbSerial by felHR85 \cite{felHR85}.
\end{itemize}
Because of similar names of projects, they will be referenced as Host API,
mik3y and felHR85.

Each method will be implemented, and tested on both smartphones with both
solutions on MCU's side.
Few test messages with different size will be send, and received echo will be
analyzed.
Selected sizes and messages are:
\begin{enumerate}
  \item 1 character:\\
  	any single letter,
  \item 32 characters:\\
  	``String with 32 charsBA987654321!'',
  \item 50 characters:\\
  	``String with 50 chars..RQPONMLKJIHGFEDCBA987654321!'',
  \item 64 characters:\\
  	``String with 64 chars................RQPONMLKJIHGFEDCBA987654321!'',
  \item 70 characters (optionally, if previous works):\\
  	``String with 70 chars......................RQPONMLKJIHGFEDCBA987654321!''
\end{enumerate}
Because buffer in MCU is 64 elements long, it should allow for good analysis of
correctness of result and performance.
MCU uses C-strings, where last element of string is ``\textbackslash 0'', so it
can actually fit only 63 characters, therefore 64 letters long string should be
too long for it.
Also, echo is in format ``\emph{protocol}: \emph{text}\textbackslash n'', where
\emph{protocol} is ``uart'' or ``cdc'', and \emph{text} is content of protocol's
buffer in MCU.
All methods are transmitting bytes, and not letters, but it's easy to transform
it in both directions.

Tests for each combination were performed, and are documented (with conclusions
in captions) on screenshots (img.
\ref{img:comm/bulk/bulk_1}-\ref{img:comm/sync/sync_uart_70_2}) in section
dedicated to each method.
However, screenshots were made only if such combination was working, and only
for Moto G (results for Xperia Neo were generally the same, if not stated
otherwise).

Before testing own implementation, tests with terminals available on Google Play
were performed, and showed, that Xperia Neo works only with CDC, and not UART,
while Moto G works with both.
It should be tested, if this is true also for own applications.

\section{USB Host API}
First method is USB Host API, which is built-in Android. 
It's a~low-level method, and because of that, it's hard to configure.
At first, it required finding endpoints which can be used for writing and
reading, which is realized by many nested-loops.
Then, usage of controlTransfer(\ldots) method is required, to configure.
It's not documented, what actually should be passed to it.
Fortunately, example working with another Freescale microcontroller was found,
and was also working with FRDM KL26Z, so it was possible to test it, however it
was working only with CDC, and not UART.

Sample code for sending and receiving message is shown on listing
\ref{lst:usbhostapi}.
\lstinputlisting[float, label=lst:usbhostapi, caption=USB Host API]
{../code_samples/bulktransfer_sample.java}
It's synchronous method, so programmer has to care for regular calling of
reading function.
In case of echo, it can be done only after sending something, but it can't be
used, if MCU sends something unexpected, like message about error, or current
state of a~robot.
Which is even worse, with this type of communication, write causes freeze of
sender, until other device reads it.
On the other hand, it means, that message can't be missed.

\includeimage{comm/bulk/bulk_1}{USB Host API - sending 1 letter}
[\\Received text is correct. 
As it can be seen, in this test it takes about 22 ms to get an echo of each sent
message.
Although, during other (not documented with screenshot) tests, sometimes (but
less often) average duration was about 12 ms.]

\includeimage{comm/bulk/bulk_32}{USB Host API - sending 32 letters}
[\\Received text is correct. 
Average time of execution was 12 ms, but in other tests, it also happened (but
less often) to be 22 ms.]

\includeimage{comm/bulk/bulk_50}{USB Host API - sending 50 letters}
[\\Received text is correct. 
Average time of execution was 12 ms, but in opposition to sending 1 or 32
letters, other values were not observed.]

\includeimage{comm/bulk/bulk_64}{USB Host API - sending 64 letters}
[\\Received text is NOT correct. 
Only first message is received completely, but sent from MCU as two.
Next ones are missing last character, and have last non-missed character from
previous one at the beginning.]

\clearpage

\section{USB Serial by mik3y}
Next method is USB Serial library available on Github of user named mik3y.
It's seems to be quite popular - it's often mentioned on StackOverflow, and on
project's page, there is a~list of many project using it.
Example usage of library is shown on listing \ref{lst:mik3y}.
\lstinputlisting[float, label=lst:mik3y, caption=mik3y]
{../code_samples/mik3y_sample.java}
Analyzing implementation of methods shows, that it makes usage of USB Host API,
but in more user-friendly, high-level form.

Unfortunately, it seems to be not compatible with chosen MCU, on any phone, and
neither on CDC, nor UART protocol.
Actually, during many tries, it was possible to send something using it, but
received text was not the one, which was send. 
Even more interesting is fact, that MCU received some data, when reading method
was called.

Therefore, this library couldn't be tested further, and also any sensible
screenshot wasn't made.
Nonetheless, it should work fine with other microcontrollers, and be easier to
use than USB Host API.

\clearpage

\section{USB Serial by felHR85}
Last found library is USB Serial, also available on Github, on account called
felHR85.
It also has some popularity, but smaller that USB Serial by mik3y.
Library allows for asynchronous transmission, with optional synchronous mode.
Another advantage is fact, that library's page mentions using \url{jipack.io}
for easier adding it to project, and also example project works with KL26Z
out-of-the-box.

\subsection{Asynchronous mode}
Asynchronous mode (listing \ref{lst:async}) offers one big advantage: it reads
using callbacks.
Programmer doesn't have to worry about continuous reading, because it's done
automatically, and only have to implement function, which operates on received
bytes.
Also, implementation of writing and reading is not using USB Host API, or at
least it much ``deeper''.

First tests showed, that it works both with CDC and UART, although as it was
expected, the latter was working only on Moto G.

\lstinputlisting[float, label=lst:async, caption=felHR85 async]
{../code_samples/felhr85async_sample.java}

\includeimage{comm/async/async_cdc_1}{felHR85, async-CDC, 1 letter}
[\\Received text is correct, but in most cases, two messages were sent at once.
Average execution time was 11 ms.]

\includeimage{comm/async/async_cdc_32}{felHR85, async-CDC, 32 letters}
[\\Received text is correct. 
In this case, all messages were sent separately.
Average execution time was also 11 ms.]

% TODO test for 31 long?

\includeimage{comm/async/async_cdc_50}{felHR85, async-CDC, 50 letters}
[\\Received text is correct, and all messages were sent separately.
Each text is divided into two lines, because it doesn't fit the window, and
Android does that automatically.
Average execution time was 12 ms.]

\includeimage{comm/async/async_cdc_64}{felHR85, async-CDC, 64 letters}
[\\Received text is  NOT correct.
Result looks the same as for using USB Host API.]

\includeimage{comm/async/async_cdc_70}{felHR85, async-CDC, 70 letters}
[\\Received text is also not correct.
It can be also seen, that the same as for 64 chars long message, ending is
missing, and next message starts with first of missing letters.]

\includeimage{comm/async/async_uart_1}{felHR85, async-UART, 1 letter}
[\\Received text was correct, but send as only a~few messages. 
In async-CDC it was only 2 messages at once, in this case all is send in 2-3.
Average execution time was about 1.5 ms, which is really faster than using CDC.]

\includeimage{comm/async/async_uart_32}{felHR85, async-UART, 32 letters}
[\\Received text looks messy, but is also regular, so it still can be seen,
that is correct. ``uart`` can be seen only once, so it means, that MCU can
continuously get text from buffer, before escaping loop.
Average execution time was about 34 ms, which is longer than using CDC, but
seems to be correlated to length of message.]

\includeimage{comm/async/async_uart_50}{felHR85, async-UART, 50 letters}
[\\Received text is correct, and looks neater than the one for 32 chars, but
it's only because of displaying it. 
As for 32 chars, ``uart: `` can be seen only from time to time (next one was
off the screen).
Average execution time was about 52 ms.]

\includeimage{comm/async/async_uart_64}{felHR85, async-UART, 64 letters}
[\\It's the first time, that 64 long message was transmitted completely without
any errors.
Average execution time was about 67 ms.]

\includeimage{comm/async/async_uart_70}{felHR85, async-UART, 70 letters}
[\\Because 64 chars long message was transmitted without errors, it was worth
checking, if even longer messages could be send. 
And indeed, with UART it's possible to do it without errors.
Average execution time was about 73 ms, so it actually increases with length.]

\clearpage

\subsection{Synchronous mode}
USB Serial by felHR85 supports also synchronous way, where programmer has to
care for reading.
In this approach, writing and reading is actually only wrapping USB Host API,
but looks much more nicely (listing \ref{lst:sync}).
\lstinputlisting[float, label=lst:sync, caption=felHR85 sync]
{../code_samples/felhr85sync_sample.java}

What is interesting, is that it worked not only with CDC, but also with UART
(although only on Moto G).

\includeimage{comm/sync/sync_cdc_1}{felHR85, sync-CDC, 1 letter}
[\\Received text is correct.
Average execution time was about 22 ms, and in opposition to USB Host API, much
shorter time was not observed.]

\includeimage{comm/sync/sync_cdc_32}{felHR85, sync-CDC, 32 letters}
[\\Same as earlier, result is correct.
What is different to sending 1 char or to USB Host API, average execution time
was always 12 ms.]

\includeimage{comm/sync/sync_cdc_50}{felHR85, sync-CDC, 50 letters}
[\\In this case, results were exactly the same as for USB Host API: text was
correct and average execution time was about 12 ms.]

\includeimage{comm/sync/sync_cdc_64}{felHR85, sync-CDC, 64 letters}
[\\Also in case of exceeded buffer, result is the same as with USB Host API:
chars exceeding buffer are missing, and one char from previous message appears
in next one.]

\includeimage{comm/sync/sync_uart_1}{felHR85, sync-UART, 1 letter}
[\\Result looks similar to one for async-CDC (so it differs on both sides) but
here more than two messages were sent at once. 
Observed average time is about 9 ms.]

\includeimage{comm/sync/sync_uart_32}{felHR85, sync-UART, 32 letters}
[\\Observed result is most similar to one for async-UART.
Result is correct, and also average time of execution was about 33 ms.]

\includeimage{comm/sync/sync_uart_50}{felHR85, sync-UART, 50 letters}
[\\This example is interesting, because text is shorter than buffer, and
result is wrong. Worth noticing is fact, that ``uart'' text can be observed
really often, and also that correct message appeared few times.
Average execution time was still related to length of message: about 57 ms.]

\includeimage{comm/sync/sync_uart_64}{felHR85, sync-UART, 64 letters}
[\\Errors for 50 chars suggest, that longer messages should also have them,
but\ldots\;result is correct!
Average execution time wasn't surprising: about 67 ms.]

\includeimage{comm/sync/sync_uart_70_2}{felHR85, sync-UART, 70 letters}
[\\This result is also surprising: errors started appearing again, but most of
messages are correct.
Actually, only 2 of errors are visible on first glance (it even looks better
than badly formatted, but correct previous ones), but there's more of them.
However, result is still better than for 50 chars, but of course it takes
longer to send it: about 80 ms.]

\clearpage

\subsection{Division of messages}
As it was seen, MCU not always receives one message at once:
in async-CDC example, few messages were sent together, and in (a)sync-CDC
exceeding buffer, message was broken into two parts.

It's also worth testing, how smartphone receives messages from MCU.
It was visualized by putting a~semicolon on end of each received block.

Results with conclusions can be seen on images
\ref{img:comm/div/div_async_cdc}-\ref{img:comm/div/div_sync_uart}.
As it can be seen, CDC transmits data in blocks (with size not bigger than
buffer), and UART does this letter by letter (and buffer is used for queuing
them).

\includeimage{comm/div/div_async_cdc}{Division of message: async-CDC}
[\\Text is received in one message. 
Semicolon is visible on line with next message, because MCU sends new line
character (``\textbackslash n''), and semicolor is added on Android side after that.]

\includeimage{comm/div/div_async_cdc2}{Division of message: async-CDC with too
long message}
[\\It can be seen, what happens when message exceeds buffer.
Each message is received as two:
64 chars long, and 5 chars long (``5432'' + `` \textbackslash n'').]

\includeimage{comm/div/div_async_uart}{Division of message: async-UART}
[\\Message is usually send letter by letter, with few exceptions.
As it was noted earlier, ``uart'' occurs only from time to time, which means,
that MCU has something to read in buffer all the time.]

\includeimage{comm/div/div_sync_uart}{Division of message: sync-UART}
[\\MCU still reads and sends message char-by-char, but smartphone reads them in
blocks. 
It can also be seen, that all blocks are correct, and errors can be
seen only between them. 
Also in this case, ``uart`` can be seen only from time to time, but more
frequently than in ``async'' version.]

% TODO sync-cdc?

\clearpage

\section{Summary}
Tests have shown, that it's possible to communicate between chosen smartphones
and MCU using two libraries: USB Host API and USB Serial by felHR85.

Consolidated performance results are shown in table below:
\begin{center}
\begin{tabular}{r|r|r|r|r|r|r|r|r|r|r|r}
& \multicolumn{3}{c|}{Xperia Neo - CDC} & \multicolumn{3}{c|}{Moto G - CDC} &
\multicolumn{5}{c}{Moto G - UART} \\
message length & 1 & 32 & \hspace{3ex}50 & 
1 & 32 & \hspace{3ex}50 & \hspace{1ex}1 & 
32 & 50 & 64 & 70
\\
\hline
USB Host API & 
11/22 & 12/22 & 12 & 11/22 & 12/22 & 12 &
\multicolumn{5}{c}{\cellcolor{red!50}not working}
\\
felHR85 - async & 
11 & 11 & 12 & 11 & 11 & 12 & 2 & 34 & 52 & 67 & 73 \\
felHR85 - sync & 
25 & 21 & 24 & 22 & 12 & 12 &
9 & 33 & \cellcolor{yellow!50}57 & 66 & \cellcolor{yellow!50}80
\end{tabular}
\end{center}

As it can be seen CDC offers the same performance for all libraries,
smartphones, and message length, because it transmits whole buffer at once.
Interesting is fact, that sometimes shorter message needed more time than
longer.
It limits message length to size of a~buffer, but message is always correct.
Therefore, if there's a~need to send something longer, it should be divided into
shorter messages, and then merged on the other device.

In case of using UART, longer messages can be sent, because they are send
byte-by-byte anyway, and buffer is using for queueing them.
Because of that, in UART there's a~need to get received message from smaller
blocks, so using some marker of message end could be a~good solution.
Another good solution could be sending some checksums, because errors were
observed during tests (but only for using sync mode with it).

Big disadvantage of UART is also lack of compatibility with older smartphones,
which is a~problem especially because CDC on MCU is a~much more complicated
solution, and not every MCU supports it, or even have USB port.
Advantage is fact, that UART is much more popular, and USB is only one of form,
it appears on microcontrollers.
However, doing it though USB openSDA port blocks possibility to use debugger on
the same port.

Each found library has some advantages and disadvantages, which can be presented
as a~list:
\begin{itemize}
  \item USB Host API:
  	\begin{itemize}
  	  \item[$+$] works with good results,
  	  \item[$+$] no external libraries,
  	  \item[$+$] low-level, many options for configuration,
  	  \item[$-$] low-level, hard configuration,
  	  \item[$-$] doesn't work with UART,
  	  \item[$-$] only synchronous approach
  	\end{itemize}
  \item USB Serial by mik3y:
	\begin{itemize}
  	  \item[$+$] popularity,
  	  \item[$+$] should work with UART,
  	  \item[$+$] high-level,
  	  \item[$-$] only synchronous approach,
  	  \item[$-$] external library,
  	  \item[$-$] not compatible with tested MCU,
  	\end{itemize}
  \item USB Serial by felHR85:
	\begin{itemize}
	  \item[$+$] the same code works both CDC and UART,
	  \item[$+$] easy to use,
	  \item[$+$] reading using callbacks in async mode,
  	  \item[$-$] external library (but can be easily added using
  	  	\url{jitpack.io}),
  	  \item[$-$] was harder to find than USB Serial by mik3y
  	  \item[$-$] UART doesn't work on Xperia Neo (but problem is probably on
  	  	phone's side).
  	\end{itemize}
\end{itemize}
Advantages and disadvantages of both protocols:
\begin{itemize}
  \item CDC:
  	\begin{itemize}
  	  \item[$+$] compatibility with both smartphones, 
  	  \item[$+$] in KL26Z, openSDA port is still available for debugging, 
  	  \item[$+$] whole message sent at once, 
  	  \item[$+$] waits, until seconds device reads message,  
  	  \item[$-$] freezes, until seconds device reads message, 
  	  \item[$-$] limited length of message,
  	  \item[$-$] requires USB in microcontroller, 
  	  \item[$-$] complex implementation in microcontroller, 
  	\end{itemize}
  \item UART:
  	\begin{itemize}
  	  \item[$+$] popularity in microcontrollers,
  	  \item[$+$] simple implementation in microcontroller,
  	  \item[$+$] length of message is not limited,
  	  \item[$+$] doesn't freeze, when message is not received,
  	  \item[$-$] limited compatibility with smartphones
  	  \item[$-$] in KL26Z, needs openSDA port for connection through USB cable,
  	 	so debugger can't be connected using USB
  	\end{itemize}
\end{itemize}

Summing up, there is several ways to communicate between smartphone and
microcontroller, which allows for using smartphone as high-level controller.
Next step is to check, if this connection can actually allow for extending
robot's functionality by something not available (or working worse) for
microcontrollers.
