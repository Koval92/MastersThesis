\chapter{Communication}

\section{Introduction}
Three ways to communicate over USB were found:
\begin{itemize}
  \item USB Host API \cite{android_reference},
  \item usb-serial-for-android library by mik3y \cite{mik3y},
  \item UsbSerial by felHR85 \cite{felHR85}.
\end{itemize}
Because of similar names of projects, they will be referenced as Host API,
mik3y and felHR85.

% TODO all screenshots on xperia

\section{USB Host API}

\lstinputlisting[float, label=lst:usbhostapi, caption=USB Host API]
{../code_samples/bulktransfer_sample.java}

\includeimage{comm/bulk/bulk_1}{USB Host API - sending 1 letter}
[\\Received text is correct. 
As it can be seen, in this test it takes about 22 ms to get an echo of each sent
message.
Although, during other (not documented with screenshot) tests, sometimes (but
less often) average duration was about 12 ms.]

\includeimage{comm/bulk/bulk_32}{USB Host API - sending 32 letters}
[\\Received text is correct. 
Average time of execution was 12 ms, but in other tests, it also happened (but
less often) to be 22 ms.]

\includeimage{comm/bulk/bulk_50}{USB Host API - sending 50 letters}
[\\Received text is correct. 
Average time of execution was 12 ms, but in opposition to sending 1 or 32
letters, other values were not observed.]

\includeimage{comm/bulk/bulk_64}{USB Host API - sending 64 letters}
[\\Received text is NOT correct. 
Only first message is received completely, but sent from MCU as two.
Next ones are missing last character, and have last non-missed character from
previous one at the beginning.]

\section{USB Serial by mik3y}

\lstinputlisting[float, label=lst:mik3y, caption=mik3y]
{../code_samples/mik3y_sample.java}

\section{USB Serial by felHR85}

\subsection{Asynchronous mode}

\lstinputlisting[float, label=lst:async, caption=felHR85 async]
{../code_samples/felhr85async_sample.java}

\includeimage{comm/async/async_cdc_1}{felHR85, async-cdc, 1 letter}
[]

\includeimage{comm/async/async_cdc_32}{felHR85, async-cdc, 32 letters}
[]

\includeimage{comm/async/async_cdc_50}{felHR85, async-cdc, 50 letters}
[]

\includeimage{comm/async/async_cdc_64}{felHR85, async-cdc, 64 letters}
[]

\includeimage{comm/async/async_cdc_70}{felHR85, async-cdc, 70 letters}
[]

\includeimage{comm/async/async_uart_1}{felHR85, async-uart, 1 letter}
[]

\includeimage{comm/async/async_uart_32}{felHR85, async-uart, 32 letters}
[]

\includeimage{comm/async/async_uart_50}{felHR85, async-uart, 50 letters}
[]

\includeimage{comm/async/async_uart_64}{felHR85, async-uart, 64 letters}
[]

\includeimage{comm/async/async_uart_70}{felHR85, async-uart, 70 letters}
[]

\clearpage

\subsection{Synchronous mode}

\lstinputlisting[float, label=lst:sync, caption=felHR85 sync]
{../code_samples/felhr85sync_sample.java}

\includeimage{comm/sync/sync_cdc_1}{felHR85, sync-cdc, 1 letter}
[]

\includeimage{comm/sync/sync_cdc_32}{felHR85, sync-cdc, 32 letters}
[]

\includeimage{comm/sync/sync_cdc_50}{felHR85, sync-cdc, 50 letters}
[]

\includeimage{comm/sync/sync_cdc_64}{felHR85, sync-cdc, 64 letters}
[]

\includeimage{comm/sync/sync_uart_1}{felHR85, sync-uart, 1 letter}
[]

\includeimage{comm/sync/sync_uart_32}{felHR85, sync-uart, 32 letters (1)}
[]

\includeimage{comm/sync/sync_uart_32_2}{felHR85, sync-uart, 32 letters (2)}
[]

\includeimage{comm/sync/sync_uart_50}{felHR85, sync-uart, 50 letters (1)}
[]

\includeimage{comm/sync/sync_uart_50_2}{felHR85, sync-uart, 50 letters (2)}
[]

\includeimage{comm/sync/sync_uart_64}{felHR85, sync-uart, 64 letters}
[]

\includeimage{comm/sync/sync_uart_70}{felHR85, sync-uart, 70 letters (1)}
[]

\includeimage{comm/sync/sync_uart_70_2}{felHR85, sync-uart, 70 letters (2)}
[]

\clearpage

\subsection{Division of messages}
% TODO division of messages

\includeimage{comm/div/div_async_cdc}{Division of message: async-cdc}
[\\Text is received in one message. 
Semicolon is visible on line with next message, because MCU sends new line
character (``\textbackslash n''), and semicolor is added on Android side after that.]

\includeimage{comm/div/div_async_cdc2}{Division of message: async-cdc with too
long message}
[\\It can be seen, what happens when message exceeds buffer.
Each message is received as two:
64 chars long, and 5 chars long (``5432'' + `` \textbackslash n'').]

\includeimage{comm/div/div_async_uart}{Division of message: async-uart}
[\\Message is usually send letter by letter, with few exceptions.
Interesting is fact, that ``uart: '' string is received only at the
beginning (and sometimes later, not visible on screen), which means, that it's
in reading loop the whole time - empty input stream should cause sending ``
\textbackslash n'' and ``uart: ``.]

\includeimage{comm/div/div_sync_uart}{Division of message: sync-uart}
[\\MCU still reads and sends message char-by-char, but smartphone reads them in
blocks. 
It can also be seen, that all blocks are correct, and errors can be
seen only between them. 
Also in this case, ``uart: `` can be seen only from time to time, but more
frequently than in ``sync'' version.]

% TODO sync-cdc?

\section{Summary}
% TODO pros/cons comparison

\begin{center}
\begin{tabular}{r|r|r|r|r|r|r|r|r|r|r}
& \multicolumn{3}{c|}{Xperia Neo - CDC} & \multicolumn{3}{c|}{Moto G - CDC} &
\multicolumn{4}{c}{Moto G - UART} \\
message length & 1 & 32 & \hspace{3ex}50 & 
1 & 32 & \hspace{3ex}50 & \hspace{1ex}1 & 
32 & 50 & 64
\\
\hline
USB Host API & 
11/22 & 12/22 & 12 & 11/22 & 12/22 & 12 &
\multicolumn{4}{c}{\cellcolor{red!50}not working}
\\
felHR85 - async & 
11 & 11 & 12 & 11 & 11 & 12 & 2 & 34 & \cellcolor{yellow!50}52 & 67 \\
felHR85 - sync & 
25 & 21 & 24 & 22 & 12 & 12 & 4 & 33 & \cellcolor{yellow!50}57 & 66
\end{tabular}
\end{center}