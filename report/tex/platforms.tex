\chapter{Platforms}

\section{Android}
Today Android is most popular mobile operating system.
Almost everyone has one or even more phones with it, so they can easily be
temporarily used as controllers.
For this thesis two smartphones were used: currently exploited Motorola Moto G
LTE (img. \ref{img:motog}), and older Sony Ericsson Xperia Neo (img.
\ref{img:xperianeo}).
\includeimage{motog}{Motorola Moto G LTE}
\includeimage{xperianeo}{Sony Ericsson Xperia Neo}

Their most important parameters are:
\begin{enumerate}
  \item Xperia Neo:
  	\begin{itemize}
  	  \item Release date: end of 2011,
  	  \item Android version: 4.0.4 (API 15),
  	  \item CPU: 1x1.00GHz,
  	  \item GPU: Adreno 205,
  	  \item RAM: 512MB
  	  \item Camera: 0.3Mpx (front), 8.1Mpx (rear),
  	  \item Screen: 3,7'' 480x854,
  	\end{itemize}
  \item Moto G (LTE):
  	\begin{itemize}
  	  \item Release date: middle 2014,
  	  \item Android version: 5.1 (API 22),
  	  \item CPU: 4x1.20GHz,
  	  \item GPU: Adreno 305,
  	  \item RAM: 1GB
  	  \item Camera: 1.3Mpx (front), 5Mpx (rear),
  	  \item Screen: 4,5'' 720x1280,
  	\end{itemize}
\end{enumerate}

Applications for Android can be written in Java (not only), which allows for
high-level programming.
Earlier, development ws done using ADT (Android Development Tools) plugin for
Eclipse, but now it was replaced by Android Studio (img.
\ref{img:androidstudio}) based on IntelliJ IDEA.
\includeimage{androidstudio}{Android Studio}
It is a~great tools, which makes writing code, refactoring, designing layout and
building application (using Gradle) really easy.
Unfortunately, it's still new, and many tutorials and are still available only
for ADT.
It's especially big problem for development with NDK (Native Development Kit),
which allows for writing code in C++, and is offering better performance, but is
still not 100\% supported by Android Studio.

However, it's a~problem only with setting up project (which can often be done
automatically by Android Studio), and all code remains the same, so it's
possible to easily find answers to most problems.
Great place for looking for them is StackOverflow, but there's also a~lot of
blogs and sites dedicated to development on Android.

Next advantage of Android is that usually the same application can be used
without any changes on different smartphones (as long as they are new enough).

\clearpage

\section{MCU}
\subsection{Introduction}
There's a~lot of microcontroller manufacturers, and even MCUs from the same one
can be completely different, so applications need to be designed for specific
one.
Freescale's FRDM KL25Z (and when it has broken, almost identical KL26Z, both
shown on img. \ref{img:MCUs}) was chosen, because it is quite popular,
\cite{mcu_on_eclipse} is a~great blog focused on Freescale's microcontroller and KL25Z in particular, and development, deployment and debugging can be done using
Eclipse-based CodeWarrior (img. \ref{img:codewarrior}).

CodeWarrior is a~great tool, which simplifies a~lot configuration of MCU's
peripherals and functions - each is presented as a~``bean''/component
configurable through GUI.
Beans with new functionality can be created or imported - lot of them are
available on \cite{mcu_on_eclipse}, with tutorials how to use them.
It's possible to auto-generate methods for beans, which allows for pseudo
high-level programming.

Most important functions of both MCUs are:
\begin{itemize}
  \item IO pins,
  \item RGB LED diode,
  \item PWM (Pulse Width Modulation),
  \item button (only KL26Z),
  \item UART,
  \item capacity slider,
  \item two USB CDC ports (one with openSDA for debugging or virtual UART),
  \item pins position compatible with Arduino platform.
\end{itemize}
Easiest method to connect them with smartphone is through USB cable using CDC or
UART.

\includeimage{MCUs}{Used microcontrollers}
[\\Freescale's FRDM KL25Z (left) and KL26Z (right)]
\includeimage{codewarrior}{CodeWarrior}

\subsection{Communication through UART}
UART (which stands for Universal Asynchronous Receiver/Transmitter) is actually
a hardware device used for sequential transmission of bits, but it's quite
popular to use its name as a~type of transmission.
It was implemented based on article 
\emph{``Tutorial: printf() and ``Hello World!" with the Freedom KL25Z Board''}
available on \cite{mcu_on_eclipse}.
MCU receives data and sends it back as long as there is something in its read
buffer (which is actually a~queue).

UART is rather popular among microcontrollers, and KL25/26Z even have it
virtualized in their USB openSDA port, so it requires only a~popular mini USB
cable. 
Unfortunately this blocks openSDA port for debugger.

\subsection{Communication through CDC}
USB CDC stands for USB Communications Device Class, and is type of device, which
can communicate using USB.
There is also an article on \cite{mcu_on_eclipse}, titled
\emph{``Tutorial: USB CDC with the KL25Z Freedom Board''},
how to make MCU a~representative of CDC.

Implementation is rather complex (easy with using beans, but generates a~lot of
code), but big advantage is fact, that it can use ``normal'' USB port, and
leaves openSDA one for a~debugger.
