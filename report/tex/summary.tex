\chapter{Summary}
As it was shown in previous chapters, Android smartphone can gather some useful
data for robot using sensors, and send it to it over USB cable.
It offers also quite good performance, and is easy to implement. 
Also compatibility is great: application implemented for older phone was working
without any changes on newer, completely different one, and boost in performance
was visible.
In MCU's case, moving from KL25Z to almost identical KL26Z required few changes
in configuration.

However, using smartphone as a controller has a few disadvantages, especially if
they would be used in ``professional'' area:
\begin{enumerate}
  \item Communication seems to be immediately (<30ms), but it can be too slow
  	for some usages.
  \item Lot of performance is used by Android itself, and lot of applications
  	and services running in background.
  \item Android is a rather insecure system, with lot of viruses.
  \item Modern smartphones are all-in-one devices, so more specialized devices
  	will easily outperform them.
\end{enumerate}
Therefore, it seems to be a bad idea to e.g. send a probe controlled by a
smartphone to Mars, with viruses and music playing in background, and battery
drained by continuous trying to connect with Facebook.

But all of listed problems doesn't actually matter, if they will be used for
fun.
It would be really nice to see already pre-built robots/drones/RC cars with USB
port and API to extend their functionality.
Some of them offers remotes in form of smartphone's application, so even an API
for them could be useful.

Currently, if someone wants to control ones available in shops, can do it only
with a remote, and controlling it programatically requires building own one
almost from scratch.
It is hard and time consuming, while it could be done much simpler and faster,
if there would be available pre-built ones.

