\chapter{Sensors}

\section{Introduction}
Modern smartphones have many sensors, and most of them can extend robot's
functionality.
Sensors differ between phones, and new (or more advanced) ones can
be connected using possible connections (mostly USB and Bluetooth).
Most popular ones are:
\begin{itemize}
  \item touch screen,
  \item accelerometer,
  \item gyroscope,
  \item microphone(s),
  \item front and rear camera(s),
  \item position sensors:
  \begin{itemize}
    \item GPS,
    \item multilateration based on GSM and/or WiFi,
  \end{itemize}
  \item magnetometer,
  \item light sensor,
  \item proximity sensor.
\end{itemize}
Some (mostly high-end, or specialized ones) have also sensors like electronic
compass, humidity/temperature sensors, fingerprint scanner, or even thermal
camera.

% TODO write about face detecting
Available implementations of face detection includes:
\begin{itemize}
  \item FaceDetector API,
  \item Camera API,
  \item OpenCV for Android,
  \item OpenCV NDK.
\end{itemize}

% TODO short description

\section{FaceDetector API}
% hard to obtain bitmap from camera

% TODO screenshot + code sample
\lstinputlisting[float, label=lst:fdapi, caption=Face Detector API]
{../code_samples/facedetectorapi_sample.java}

\includeimage[0.4]{facedetectorapi}{Face Detector API}

\section{Camera API}
% max 5 faces
% 30 ms in good light/on screen, 60 ms in bad
% resolution doesn't have impact on speed, only on quality
% detection in background

\lstinputlisting[float, label=lst:cameraapi, caption=Camera API]
{../code_samples/cameraapi_sample.java}

\section{OpenCV for Android}
% detection in user's code

\lstinputlisting[float, label=lst:opencv, caption=OpenCV]
{../code_samples/opencv_sample.java}

\includeimage{opencv_high}{openCV - high resolution}
[\\All faces were detected, but also false matches can be seen. Faces were
always found, and false matches occurred only from time to time.]

\includeimage{opencv_low}{openCV - low resolution}
[\\Not all faces were detected, but there's also no false matches. Frame
around face has the same width (3px) as on previous image, so resolution is
really small.]

\section{OpenCV NDK}
% real time?
\includeimage{opencv_ndk}{openCV - NDK}
[\\Application has limited screen area, so only 3 rows fitted. False matches
also occurred, and not all faces were discovered correctly, but detection was
working in almost real-time, and quality is still high.]

\section{Summary}
% TODO pros and cos of each method

Resolutions available for preview for both smartphones (obtained during
configuration needed for Camera API):
\begin{center}
\begin{tabular}{r@{ x }l|r@{ x }l}
\multicolumn{2}{c|}{Moto G} & \multicolumn{2}{c}{Xperia Neo} \\
\hline
1280 & 960 & 1600 & 1200 \\
1280 & 720 & 1280 & 720 \\
960 & 720 & 864 & 480 \\
864 & 480 & 640 & 480 \\
768 & 432 & 480 & 320 \\
720 & 480 & 352 & 288 \\
640 & 480 & 320 & 240 \\
320 & 240 & 176 & 144 \\
176 & 144 \\
\end{tabular}
\end{center}
As it can be seen, some resolutions are available for both smartphones, so they
were the ones used during testing of Camera API. 
However, they couldn't be used for other methods.
Face Detector API doesn't work on camera preview, file with the same faces was
used, and scaled to have the same height as resolutions available to Camera API. 
On the other hand, in OpenCV only maximum width and height could be set, so it
was set to the same as in Camera API, but actual resolution could be a little
smaller.

Average time needed for calculating single frame are as follows:
\begin{center}
\begin{tabular}{r@{ x }l|r|r|r|r|r|r}
\multicolumn{2}{c|}{} & \multicolumn{3}{c|}{Xperia Neo} &
\multicolumn{3}{c}{Moto G}
\\
\multicolumn{2}{c|}{} & \hspace{4ex}FD & Camera & OpenCV & \hspace{4ex}FD &
Camera & OpenCV
\\
\hline 1280 & 720 & 420 & 180 & - & 380 & 33 & 1500 \\
864 & 480 & 390 & 110 & - & 320 & 33 & 1200 \\
640 & 480 & 390 & 123 & 5000 & 320 & 33 & 1000 \\
320 & 240 & - & 128 & 2500 & - & - & 275 \\
176 & 144 & - & 101 & 500 & - & - & 105
\end{tabular}
\end{center}
